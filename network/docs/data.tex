\section{Data Transmission}

To transmit data packets from the event bus, the network stack must
acquire the variable length packets, place them into larger udp
datagrams, set the metadata fields, and transmit them. However, we
must also save a copy of each packet (for possible retransmission),
and respond to a retransmission request should one occur during the
normal stream of data transmission.

To enable retransmission we have a packet TX FIFO with space for 256
1024-byte packets; a data packet should never exceed 600 bytes. This
FIFO also gives us headroom should the TX Fifo fill due to
insufficient access to the TX Mux interface. Note that, on average, we
have more than enough TX bandwidth -- but in the short run it is
possible to fill the TX fifo and thus need to buffer packets in the
in-memory FIFO.

This interface is heavily pipelined, as data bus packet transmission
is not viewed as an event-critical process.

We have various FIFOs and synchronization interfaces to guarantee our
goal of, within an ECYCLE, being able to service:

\begin{itemize}
\item The arrival of two new data bus packets and their subsequent
  memory writes to the fifo.
\item The placement of two packets in the memory fifo into the output
  FIFO for transmission
\item The retrieval of one packet for retransmission
\end{itemize}

\subsection{Data Bus Acquire}

\subsection{Data Bus Packet Gen}

\subsection{Memory Arbitration}

\subsection{Data TX  FIFO}
