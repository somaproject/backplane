\section{Overview}

The Soma Network Stack is a high-bandwidth low-latency IP-based
network stack which implements the Soma Network Protocol and several
convenince routines. The implementation interacts with the Soma Event
Bus, the Soma Data Bus, and the Soma Network Interface.

To comply with the soma Network Protocol the Network Stack supports 
\begin{itemize}
\item Broadcast of variable-length UDP datagrams containing Data Bus data. 
\item Broadcastof variable-length UDP datagrams containing Event Bus events
\item Retransmission of lost Data Bus datagrams via unicast requests 
\item Reception of inbound event datagrams. 
\end{itemize}

and additionally provides 
\begin{itemize}
\item ARP who-has query response
\item ICMP echo-response (ping) for diagnostics
\end{itemize}

\subsection{Bandwidth}
A data packet is going to be at most 280 ticks (560 bytes) long, and a
event packet is going to be at most 550 ticks (1100 bytes) long.

This means that we can transmit one event and 1.6 data packets per
ecycle, or 50 event packets and 80 data packets per ms.


\subsection{Limitations}
The Stack does not support IP fragmentation, and has no internal ARP
cache to allow transmission of new packets to arbitrary hosts. This is
normally not a problem as all data transmission will be either in
response to an incoming packet or broadcast to all hosts on the
subnet.

\import{}{design.tex}
\import{}{implementation.tex}
\import{txmux}{txmux.tex}
\import{inputcontrol}{inputcontrol.tex}
\import{data}{data.tex}
\import{retx}{retx.tex}
\import{eventtx}{eventtx.tex}
\import{eventreceive}{eventreceive.tex}
\import{arp}{arp.tex}
\import{ping}{ping.tex}

