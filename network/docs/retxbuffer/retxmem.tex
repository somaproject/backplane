\section{ReTX Buffer Interface} 
The memory retransmission FIFO is a relatively simple interface to the
MemDDR2 module to handle the reading and writing of memory.

Since the memddr2 interface is focused on writing 1024-byte rows, one
at a time, we can store an entire packet in a row and thus have 32768
spaces for packets. We split this space up evenly into two chunks, A
and B, each of which has 16k entries (14 address bits).

For the data packets, this lets us store a maximum of 64 packets of
each possible source/type combo, for ~64 ms of buffer. For events,
this works out to ~300 ms of events.
\subsection{Interface}


The interface runs internally at 150 MHz * 32 bits == 300 MB/sec,
which means we can complete a row transaction in under 2 us -- or
roughly 10 row transactions per ECYCLE. This is more (!) than enough. 

The mapping between some sort of unique ID and buffer position is left
to the external interface. We had considered various advanced hashing
schemes but ultimately decided to simply push that decision out to the
external modules.

Our interface to submodules is modeled after existing network
interface components, and this desire to remain compatible has led to
the interface being more opaque than I would have liked. In general,
to write to the fifo the source performs random-acces writes (16-bits
wide) into a write fifo using a combination of WDIN and WADDR
lines. To signal the completion of an input packet write, WDONE is
asserted with the packet's location in the fifo indicated by WID. This
interface runs at 50 MHz, the system core clock speed.

Internally, the interface just muxes between the four round-robin. 

Because the memddr2 interface writes to a target output, we will need
to write into one of two output fifos and then provide access from
them. Current code expects the retx interface to write into a buffer
as well. This double-write strategy, while inefficient, is necessary
to provide maximum backwards compatibility, and can be rewritten
should the need arise.

Each Read and Write interface (4 in total) have their own clock. in
this way we guarantee that we can read out as fast as we like.

This guarantees, for example, that if we run the data input interface
at 100 MHz, we can always put in two data packets every ecycle.

\subsection{Implementation}
