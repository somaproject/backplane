\section{Overview}

The DDR2 Memory Interface presents a block-based interface to the DDR2
DRAM that's used as a packet buffer. Memory is transferred in blocks
of 256 4-byte words via an interface optimized for interface with FPGA
BlockRAMs.


\subsection{Interface}
The primary interface runs at 150 MHz, and requires all four 90-degree
clock phases as input. A block transfer begins with the assertion of
\signal{START} and terminates with the assertion of \signal{DONE}. The
requested row block is indicated by \signal{ROWTGT[14:0]}.

The write interface consists of \signal{WRADDR[7:0]} and
\signal{WRDATA[31:0]}. Througout the course of the transaction
\signal{WRADDR[7:0]} will change to request a data word, which must be
present on \signal{WRDATA}{31:0} two ticks later. This will most
commonly be implemented via a BlockRAM and a pipeline register.

The \signal{RDWE} signal on the read interface writes the read block
into the output RAM buffer. Again, this will be most commonly
implemented with a block ram.

\subsection{Implementation} 
DRAM's complexity is greater than NoBL/ZBT SRAM due to its
burst-orientated nature and need for bank precharge, activation, and
refresh commands. DDR2 DRAM exacerbates these challenges by both using
double-data-rate transfer and being fundamentally source-synchronous.

The DRAM control signals are controlled by a large control multiplexer, which selects between the following modules for activity:

\begin{itemize}
\item Boot Configuration: Handles start-up, register writing, and configuration of the RAM. 
\item Refresh Module: Provides the periodic refresh command and delay required by the DRAM. 
\item Write Module: Performs a 1024-byte burst write to a given DRAM row. 
\item REad Module: reads a 1024-byte burst from the DRAM and writes it to the output buffer. 
\end{itemize}

In addition, we use two DQAlign modules to delay the input data from
the RAM DQ lines to be properly aligned with our internal
\signal{CLK}. This is done using the Xilinx IOB Delay element and
locking on to the input DQSL/DQSH signals.

