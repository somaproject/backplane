
\section{Event Port Interface}

Event routeing is based around the EventBus Port interface. This is a
curious interface as it requires partial cooperation with the client
devices for timing, thus violating an abstraction barrier. However,
such is the price of embedded hardware design.


The eventbus port communicates to a target internal device via five signals:

\signal{EARX[N-1:0]} Event address receive: constant from the
beginning of ECYCLE, this contains the list of targets for the next
address cycle. the assertion of EARX[m] on EventBus Port N means that
the device connected to port N transmitted an event targeted for
device m.

\signal{EDRX[7:0]} Event data input. 

\signal{EDSELRX[3:0]} Event data mux selection, selects the relevant
byte in the pending event data.

\signal{EATX[N-1:0]} Output event enable. If EATX[m] is asserted on
eventbus port N, then the device connected to port N is the target
(i.e. should respond to) the m-th event in the event data block.

\signal{EDTX} output event data, in a stream. This signal is common to all ports, that is, all ports send the same data at once (event target specificity is determined by EATX)

On the cycle following ECYCLE, after 48 ticks into the event cycle
(that is, the cycle where ECYCLE = '1' counts as tick 0, and there are
47 ticks after that, then this tick) the event router begins
transmitting the data from the device on EventBus Port 0.

THIS NEEDS MORE EXPLANATION! 
Why do we wait? because that's how all implementations will work. Sure, it's a layer 2/3 mix, but whatever. 


\section{Implementation}

The implementation is fairly trivial, using a cascade of counters to
meet the apropriate multiplexing and timing requirements; note that
outside of the above-defined timing parameters, the behavior of the
router is undefined.

