

This prompts us to formalize our serial protocol. There's a problem
because our serial protocol is asynchronous. We'll make the assumption
that our clocks are close-enough that we can just send our data 5x
slower, and successfuly recover the bits.


Our bitstream is a list of 5 bits:


\section{bootserialize}

0
FPROG
FCLK
FDIN

all other bits are ones. 

Capture is initiated by the 1->0 transition, which prompts the capture
of the next 4*5 = 25 bits; the middle bit is sampled.

BCNT just counts to 4 and resets

This is a very simple FSM which just uses the BCNT as a clock-enable
and outputs each bit for 5 ticks.


\section{mmcfpgaboot}

The MMCFPGAboot interface lets you give it an address, a length, and a
list of devices, and it will serial-boot those devices.

upon assertion of \signal{START}, the FSM first serializes an
assertion of the FPGA's \signal{PROGRAM} line for a fixed duration to
reset the target chips.

Then, we use the MMCio interface (defined elsewhere) to read in a
single block from the MMC interface.

We read the full 512 bytes into a buffer and then use an additional
counter on the other side of the BlockRAM to read them out a bit at a
time. 

We serialize these via the BootSerialize module, asserting the fclk as
we go along.

Wen we have transfered enough blocks, MCNT will equal bootlen, and we
stop and assert \signal{DONE}.

