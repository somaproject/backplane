\subsection{Boot Storage SPI Interface}

To talk to the boot storage device over SPI, we use a somewhat
convoluted serial SPI interface. Having the backplane core FPGA be the
master would have been nice, but the target boot storage device, in this
case an 8051, would make a poor SPI slave with our bit-banging SPI IF. 

The FPGA can request an SPI read through the assertion of the
BOOTSPIREQ line, which will then cause the master to query for a
request.

FPGA requests: At the moment, there are basically two requests, ``open
file'' and ``read bytes''.

8051 Commands: The commands the 8051 can send are: 
0x00: no op
0x01: query request
FOCMD: 
\subsubsection{File Open Command}
MOSI:          MISO:
CMDREQ          - 
 0x00           -
  -            FOCMD
  -            0x00
  -            Filename[0]
  -             ...
  -            Filename[31]
     [there might be a substantial pause here]
FOPENRES         -
0x00 ??            -
FOPENLEN[3]      -
...
FOPENLEN[0]      - 
Done!


CMDREQ = 0x01
FOCMD  = 0x01
filename: the null-terminated filename string to open
FOPENRES = 0x01 success
           0x02 file not found
           0x03 other error
and then either all 0s or the file length in bytes


\subsection{File Read Command}
MOSI:           MISO:
CMDREQ            -
0x00              - 
  -             FRCMD
  -             0x00
  -              OS[3]
  -              ...
  -              OS[0]
  -             LEN[3]
  -             ... 
  -             LEN[0]
   [there might be a substantial pause here]
BYTE[0]           -
 ...              as well as substantial block-related delays
BYTE[N]           - 
  
CMDREQ: 0x02
FRCMD : 0x02
OS: the offset into the file
LEN : the number of bytes to read
Byte0[0] the first byte in the file
ByteN: the last byte requested

Note our guarantee: We always return N bytes, and pad with 0s


\subsubsection{IO Interface} 
We require the higher-level interface to understand the per-byte
specifics of this IO, but we do wrap
all of the SPI stuff in the BootSPIIO module which has: 

DIN[15:0]
DOUT[15:0]
ADDR[10:0]
WE 
CMDREQ: request a command be process
CMDST : asserted when the command begins
CMDDONE: asserted when the command is done
CURBYTE[10:0] the currently-processed BYTE

The SPI side actually runs at 150 MHz to handle the SPI IF from the
device without having to introduce transfer-rate-killing wait states.
We assume that the maximum SPI clock rate on the 8051 side is 25 MHz.

Note that with DOUT[15:0], DOUT[15] is placed on the wire first, and
DOUT[0] last; similarly, DIN[15] was the first bit placed on the wire
by the master.

