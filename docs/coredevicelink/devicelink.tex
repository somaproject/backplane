Notes and design on devicelink implementation

The devicelink implementation abstracts away bit serialization and reception via the virtex-4 serdes components. 

\subsection{Deserialize}
The deserialization hardware is a fairly straight-forward implementation of the Xilinx. Note the output of the serialized bits is such that Q1 corresponds to \signal{DOUT(9)}. 

\subsection{Serialize}
Serialization is more difficult as the hardware serialization hardware
can only handle 1:10 serialization, but to incorporate clock framing
we need to perform 1:12 serialization. Thus our serialize module takes
in 50 megabyte/50 MHz data and performs the necessary conversion,
adding clock frame bits and sending the proper subset of bits out in
each cycle at 60 MHz.


\subsection{FSM} 
The Core Devicelink FSM can be broken into five segments

\subsubsection{Force Drop}
Upon return to the none state, we attempt to forcibly drop the
connection; we do this by sending only zeros for 4 ms with no clock
frames by asserting \signal{STOPTX}.

\subsubsection{Send Sync Pulse}
We then send a series of clock-framed zeros to let the receiving
device lock for another 4 ms.

\subsection{Bit (eye) Align}
This is the most difficult part of the interface; we use the IDELAY
adjustable input delay in the Virtex-4 input buffer. We attempt to
increment the delay three times without changing the received word
(that is, keeping \signal{RXWORDL} = \signal{RXWORDLL}). The logic
here is that if we can increment the delay N times without seeing
a change in the bit pattern, we're in a ``stable'' region of the eye, and
so we center on that region. 

The FSM starts in BITINC, increments the delay, and then waits for 64
ticks, making sure it receives the same word that entire time. If so,
we transition to BITGOOD and increment BITGOODCNT. When Assuming three
successful delay increments without changing the received word, we
transition to \state{BITBACKUP} and create a delay decrement to center
ourselves on the eye of the received words.

Should this process take longer than 20 microseconds, we assume the
link is invalid and transition back to \state{NONE}.

\subsection{Lock to Word}
The FSM then bit-shifts the input word looking for either 0x343 or
0x0bc (8b/10b k28.0) , the words being transmitted by the device
during this phase.

\subsection{Data validity check}

We then attempt to read four error-free decoded bytes from the 8b/10b
decoded data. Should this succeeed, we transition to \state{SENDLOCK}
and send a single 0xFE (k30.7) character.

\subsection{Lock}

At this point we have reached the lock state. Should we receive either
an error in our 8b/10b data or an assertion of \signal{DROPLOCK}, we
transition back to \state{NONE}.
