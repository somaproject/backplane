
The DeviceLink protocol describes the mapping between events and
data on the event and data buses to series of symbols on the
devicelink. It also incorporates the transmission of the relevant
EVENT/data symbols.

The protocols involved are closely linked with the soma RX and TX
protocols. In particular, event and data broadcast are handled
here. The devicelink interface supports four client devices.

All DeviceLink TX activity is organized in a series of 1000-symbol
frames, each frame corresponding to a Soma Event Cycle. The beginning
of each frame is coincident with the beginning of an event cycle, and
is indicated by K28.5.

Event Cycles consist of broadcast messages to up to 78 different
target devices. 


\subsection{TX Link} 

we have an EVENTCE signal, internally, which gates an input event. 

The header consists of:
1. a k28.5 sync comma
2. four 11-symbol device configuration packets:
byte 0: device can send data during this cycle (DGRANT)
byte 1: lsb: the first event in the sequence is for this device
byte 2: lsb: the 8th evenet in the sequence is for this device
... through byte 10

The total header duration is 11*4 + 1 = 45 bytes. The events are
subsequently sent on the 49th byte (inclusive to the k28.5), that
is, there are 4 full-events-worth of packets before we start. 



\subsection{RX Link}

Device transmit:
every device must respond with a K28.5 at the beginning of each event cycle

A device may also send up to 550 bytes of data, if selected, in the following needlessly complex format

to begin data, send K28.2
to end data, send K28.3

Note that this requires the data be a single continuous burst. 


If the device has any events, it can send them any time during the event cycle, but it must first send the data packet. 


Any time during an event cycle a device can send an event, which in
this case is a 22-byte packet of data prefaced with a K28.4 comma
character:
  First, the 10-byte bitmask for device target selection
  Then, the 12 bytes of the data itself



