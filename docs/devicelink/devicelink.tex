
The soma backplane communicates with external devices via the Soma
DeviceLink interface.

The DeviceLink interface is a full-duplex high-speed serial connect
capable of transporting arbitrary byte streams and selected control
characters at 30 MB/sec in both directions. Links are labeled TX and
RX from the perspective of the Soma Core. The DeviceLink is
source-synchronous from the core, guaranteeing one symbol per clock
tick for both core and device.


\subsection{Wire Protocol Asymmetry}

High-speed point-to-point serial interfaces allow for the data rates
common in parallel shared busses without the PCB layout or signal
integrity issues. 

While the Soma Core can easily handle serial data streams into the
hundreds of megahertz, device FPGAs, in particular the Spartan-3, are
unable to do so. They also cannot natively perform clock recovery from
the encoded data stream. To solve this problem, devices implementing
DeviceLink can take advantage of off-the-shelf deserialization ICs made
by National Semiconductor (DS92LV1224) and Texas Instruments (TI PART
NO).

The DeviceLink consists of two serial data streams, one in each direction: 
\begin{itemize}
\item \textbf{TX Link} : The TX Link wraps serialized 8b/10b-encoded
  words in two additional clock frame bits as necessitated by low-cost
  deserializer ICs, giving a total raw bitrate of 360 Mbps. The TX
  Link's embedded clock is recovered by the device and used as the
  master clock for internal devices.
\item \textbf{RX Link} : The RX Link is simple 8b/10b-encoded serial
  data, transmitted by the device at a raw 300 Mbps. This data rate is
  easily accomplished via the DDR IOBs on a device's Spartan-3 FPGA.
\end{itemize}


8b/10b encoding was selected because: 

1. no longer patented 

2. guarantees clock transitions for our synchrnization efforts

3. widely understood 

4. has embedded punctuation to allow framing and out-of-band
   signalling with minimal effort, including K28.5, which allows for
   easy byte locking

5. Xilinx has an easy-to-use implementation -- in time, we hope to
   replace this with native IP for a truly Free Solution

\subsection{Physical Layer}
The serial streams are transmitted via low-voltage
differential-signalling (LVDS), offering superior immunity to
common-mode noise and reducing issues associated with grounding. 

DeviceLink uses LVDS pairs with 50-ohm single-ended impedance and
100-ohm differential. These are readily available in all Xilinx FPGA
families.

\subsubsection{PCB Layout} 
For high-speed serial interconnect, the SATA standard recommends "differential microstrip traces (100+/- 5 ohms) over a ground plane (single ended 50+/-2.5 ohms). However, this results in a trace width and spacing that is just unmanagable. 

According to PCBexpress, these processes are dependent on the exact
geometry of the copper between the traces.

See 4pcb.stack.txt
    pcb-express1.txt
    pcb-express-2.txt

A good number to go with is 7-8 mil, and so we're using the following:
epsilon = 4.6
w = 11 # trace width 
s = 14. # trace separation 
t = 1.4 # trace thickness 
h = 7.5 # height above ground plane

Z0 =  53.3
Zdiff =  98.1


A little high on the common-mode impedance, but this is less of an issue and the best we could do given our PCB layer stack-ups. 



\subsubsection{Device Connectors}

The SATA backplane connector will be used; SATA signal pairs have a
100-ohm impedance.

In terms of actual connectors, we have a few options: 

On the DSPboard side, we have: 
  The device plug with the strattle mount: 87679-0003 (Arrow has 1k)
  Device plug smt without jumpers -- 87703-0001 (arrow has 1k)

On the backplane side, we have two of our standard-height connectors: 
  1. 87713-1001 -- SMT mount

get part numbers and pictures



\subsection{The link lock sequence}


The source-synchronous nature of the TX Link and the need for the core to recover the RX Link stream necessitate an initial synchronization step before any DeviceLink can be considered ``up''. 


the steps of link establishment are listed below: 

1. upon detecting a dropped link, the core first suspends all data transmission for 1 ms, and then begins sending a sequence of all 0s via the TX line


2. The period of inactivity will force the device to lose lock, reset, and then enter lock initialization mode. 

3. Here it will wait until it has established a lock on the inbound data stream. At this point, it will begin broadcasting a constant sync pattern, alternating zeros and ones. 

4. the core will meanwhile be waiting a millisecond, and then  for this pattern and will use it to lock onto both bit transitions and symbol alignments
5. once the core has acquired lock, it will transmit a  K30.7 comma character to the device indicating this is the case, at whic point the device will consider the link locked as well. 

FPGA Core implementation -- IO 

Link to somabackplane/devicelink/docs for the prototypical client implementation and API 



How does the core do it's thing? 

First, vary phase to detect the transitions... what if we're not sending yet? 

