The boot control interface serves as the bridge between the output of
the rx event fifo and the mmcfpgaboot interface, handling command
recept and response.

Boot Is the simple low-level interface to the MMC-based early-stage
boot configuration for all devices. 

It receives a single command, 0x20, which is structured as follows: 


word 0: high bits of field mask
word 1: low bits of field mask

word 2: length (in 512-byte blocks) of boot sequence
word 3: address to start to boot from

it sends one response back to the original sender


0x20: Response, with : 
   byte 0: 0x0002 -- boot complete and successful
	   0x0001 -- boot request ignored, current boot pending


           The interface of bootcontrol is such that the first boot
           request it receives places it in ``booting'' mode
           (\signal{BOOTING} = 1) and all subsequent requests cause an
           error response until the boot is complete.

\section{FSMs}

On each cycle, the FSM checks for completion of a previous boot via the assertion of \signal{MMCDONEL}. If this signal is asserted, we transmit a ``boot successful'' event back to the source event and then do nothing for the remainder of the cycle. 

Otherwise, we check to see if there are any pending events in the rxeventfifo, and if we receive a valid boot event, we take two actions based on wether or not we have a current boot pending:

\begin{enumerate}
\item Current boot is pending: transition to \state{BOOTERR} and set the relevant output bit.
\item No pending boot: Trigger the boot, set the relevant bits, and begin the boot cycle. 
\end{enumerate}

The existence of \signal{LEARX} (latched-\signal{EARX} lets us respond
to multiple simultaneous (invalid) boot requests, so that if we
recived six boot requests during a single event cycle, we could
successfully boot using the first and then in the same cycle respond
to the others with an error event.
