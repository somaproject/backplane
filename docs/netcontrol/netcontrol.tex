NetControl handles all layer 3 and above control and status
information. At the moment this includes IP-settings and various
network status counters.

------------------------------------------------------------------------
Network Settings
------------------------------------------------------------------------

CMDNETWRITE : 0x42
from any address, DATA[0] says what we write:
    DATA[0] = 0x0001 : DATA[2:3] is IP address
    DATA[0] = 0x0002 : DATA[2:3] is IP bcast address
    DATA[0] = 0x0003 : DATA[2:4] is MAC address

CMDNETQUERY : 0x43

queries the current value, with DATA[0] being the requested word,
returned as above.

both of these respond with CMDNETRESPONSE, with Data[0] indicating
the word and the relevant bits as above. 

------------------------------------------------------------------------
Network Counters:
------------------------------------------------------------------------

these are all 32-bit counters that contain packet counts, with: 
CMDCNTQUERY: 0x40
CMDCNTRST : 0x41

CMDA : reset (with associated bitfield)
CMDB : query DATA[0] = requested register
   gets a response with 
   CMDB: DATA[0] = register

   also, we periodically broadcast to everyone the register values,
   say every 100 us.


--------------------------------------------
1. read event
2. act on event
3. check periodic counter
4. send if necessary
5. 
---------------------------------------

Counter read addresses / positions

0: 0x12345678 // verification
1: RXIOCRCERR count
2. RX packet count

8 : TX0LEN (byte length of TX Port 0)
...
15 : TX7LEN (byte length of TX port 7)
16 : TX0CNT (packet count of TX port 0)
...
23 : TX7CNT (packet count)
