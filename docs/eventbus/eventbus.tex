
\section{The Event Bus}

The Event Bus lies at the heart of the low-latency capabilities of the Soma system. The bus itself interconnects all low-latency Soma peripherals along a common bus, allowing for broadcast and point-to-point communication of events. The sequencing of devices on the bus and the placement of events is controled by the Event Bus Sequencer. An event requires a device to drive the bus for six clock cycles, or one event cycle. 

      Each device on the event bus has a unique event ID, and the bus supports a total of 48 such devices. Each event can be sent to an arbitrary subset of devices.  

\begin{SignalTable}{Event Bus Pinout}
\signaldef{\bus{EDATA}{15:0}}{16-bit wide tristated data bus}

\signaldef{\signal{SYSCLK}}{ system clock, used by the bus}

\signaldef{\signaln{EVENT}}{Shared line, active low, telling peripherals that a new event begins on the next clock cycle.}

\signaldef{\signaln{ECE}}{Coincident with \signaln{EVENT}, tells a device to drive the event bus during the next event cycle. Each device has its own \signaln{ECE} line; these are sequentially activated by the Event Bus Sequencer.}

\signaldef{\bus{EADDR}{7:0}}{Bit field addresses representing, over the course of an event cycle, the target device for the current command.}

\end{SignalTable}


\subsection{Event packet description}

Each event has a header, consisting of a command byte and a source byte. The second word of each event will generally contain some form of metadata, leaving the remaining four words (8 bytes) for actual data.

\begin{figure}
\includegraphics[ext=pdf]{eventbus.example.timing.pdf}
\caption{Example Event Bus Transaction.}
\end{figure}

\subsection{Timing}
The system clock runs at 20 MHz, thus each event takes 300 ns, and a full event cycle (polling all 48 devices) requires 14.4 us. This means each devices will be able to transmit roughly 70k events per second, or a bandwidth of 550 kB/second. 
